\documentclass{beamer}

% =========================
% THEME & STYLE
% =========================
\usetheme{Madrid}
\usecolortheme{default}
\useoutertheme{split}
\useinnertheme{circles}

% =========================
% PACKAGES
% =========================
\usepackage{xcolor}
\usepackage{mathrsfs}
\usepackage{amsmath, amssymb}
\usepackage{booktabs}
\usepackage{hyperref}
\usepackage{animate}
\usepackage{caption}
\usepackage{bbm}
\usepackage{bm}
\usepackage{tikz}
\usetikzlibrary{positioning, arrows.meta}


\captionsetup[figure]{font=tiny}

% =========================
% BIBLIOGRAPHY
% =========================
\usepackage[
    backend=biber,
    style=alphabetic,
    sorting=ynt
]{biblatex}
\addbibresource{bib.bib}

% =========================
% COLORS (CS / ENS)
% =========================
\definecolor{CSred}{RGB}{160,32,60}
\definecolor{CSgrey}{RGB}{136,121,150}
\definecolor{UPSred}{RGB}{97,21,58}

\setbeamercolor{titlelike}{bg=CSred}
\setbeamerfont{title}{series=\bfseries}
\setbeamercolor{palette primary}{bg=CSred,fg=white}
\setbeamercolor{palette secondary}{bg=CSred,fg=white}
\setbeamercolor{palette tertiary}{bg=CSgrey,fg=white}
\setbeamercolor{palette quaternary}{bg=CSgrey,fg=white}
\setbeamercolor{structure}{fg=CSgrey}

% =========================
% METADATA
% =========================
\title[Deep Learning]{Deep Learning}
\subtitle{Presentation}

\author[Jamal, Martini]{
Adonis~JAMAL \and Jean-Vincent~MARTINI\\
Explaining and Harnessing Adversarial Examples
}

\institute[ENS Paris-Saclay, CentraleSupélec]{
École Normale Supérieure Paris-Saclay - MVA\\
CentraleSupélec - MDS
}

\date[Jan 06, 2026]{January 6th 2026}

\titlegraphic{
    \centering
    \includegraphics[height=1.2cm]{img/logo_mva.jpg}
    \hspace{1cm}
    \includegraphics[height=1cm]{img/logo_ens-ps.png}
    \hspace{1cm}
    \includegraphics[height=1cm]{img/Logo_CentraleSupelec.svg.png}
}

\begin{document}

\frame{\titlepage}

%%%%%%%%%%%%%%%%%%%%%%%%%%%%%
% INTRODUCTION
%%%%%%%%%%%%%%%%%%%%%%%%%%%%%
\section{Introduction}
\begin{frame}{Motivation \& Problem Statement}
        \textbf{The Context:}
        \begin{itemize}
            \item Neural Networks are vulnerable to \textbf{Adversarial Examples}: inputs with worst-case perturbations causing high-confidence errors \cite{main}.
            \item This is a security risk for safety-critical applications (e.g., autonomous driving).
        \end{itemize}
        
        \begin{block}{Project Goals}
            \begin{enumerate}
            \item Analyze the seminal work by Goodfellow et al. (2015).
            \item Extend the analysis from Classification to Object Detection (YOLO11n).
        \end{enumerate}
        \end{block} 

        % \centering
        % \includegraphics[width=\linewidth]{../report/img/screen.png}
        % \footnotesize{Visualizing weight filters \cite{main}}
\end{frame}

%%%%%%%%%%%%%%%%%%%%%%%%%%%%%
% METHODOLOGY
%%%%%%%%%%%%%%%%%%%%%%%%%%%%%
\section{Methodology}
\subsection{Theoretical Framework}
\begin{frame}{Theoretical Framework: The Linearity Hypothesis}
    \textbf{Why are deep networks vulnerable?}
    \begin{itemize}
        \item \emph{Old Hypothesis:} Overfitting or extreme non-linearity.
        \item \emph{Goodfellow's Hypothesis:} Models are \textbf{"too linear"} in high-dimensional spaces.
    \end{itemize}

    \vspace{0.5cm}
    Consider a linear activation $\bm{w}^\top \bm{x}$. With perturbation $\bm{\eta}$:
    \begin{equation}
        \bm{w}^\top (\bm{x} + \bm{\eta}) = \bm{w}^\top \bm{x} + \bm{w}^\top \bm{\eta}
    \end{equation}
    
    If we set $\bm{\eta} = \varepsilon \cdot \text{sign}(\bm{w})$:
    \begin{itemize}
        \item The activation grows by $\varepsilon m n$ (where $n$ is dimensionality).
        \item Many small changes accumulate to a massive shift in output.
    \end{itemize}
\end{frame}

\subsection{Fast Gradient Sign Method}
\begin{frame}{The Fast Gradient Sign Method (FGSM)}
    FGSM is a single-shot attack designed to maximize loss under an $L_\infty$ constraint.
    
    \begin{block}{The Attack Formula}
    \begin{equation}
        \bm{\eta} = \varepsilon \cdot \text{sign}(\nabla_x J(\bm{\theta}, \bm{x}, y))
    \end{equation}
    \end{block}

    \begin{itemize}
        \item \textbf{Efficient:} Requires only one backpropagation pass.
        \item \textbf{Dual Use:} Used for generating attacks and for adversarial training (regularization).
        \item \textbf{Constraint:} $||\bm{\eta}||_{\infty} < \varepsilon$ ensures imperceptibility.
    \end{itemize}
\end{frame}


%%%%%%%%%%%%%%%%%%%%%%%%%%%%%
% RESULTS & EXTENSIONS
%%%%%%%%%%%%%%%%%%%%%%%%%%%%%
\section{Results \& Extensions}
\subsection{One-Stage Object Detection}
\begin{frame}{Our Contribution: Object Detection Setup}
    We extended the study to \textbf{One-Stage Detectors}.
    \vspace{0.3cm}
    \begin{columns}
        \column{0.005\textwidth}

        \column{0.595\textwidth}
        \textbf{Experimental Setup:}
        \begin{itemize}
            \item \textbf{Target Model:} YOLO11n (Nano) \cite{yolo11_ultralytics}.
            \item \textbf{Transfer Target:} SSDlite MobileNetV3.
            \item \textbf{Dataset:} Trained on COCO, Evaluated on VOC2007.
            \item \textbf{Metric:} Mean Average Precision (mAP) @ IoU=0.5.
        \end{itemize}

        \column{0.4\textwidth}
        \centering
        \includegraphics[width=0.9\linewidth]{../report/img/009909_0.00.jpg} \\
        \footnotesize{Clean Detection (Confidence 0.70)}
    \end{columns}
\end{frame}

\subsection{Attack Results}
\begin{frame}{Attack Results: Failure Modes}
    FGSM successfully degrades detection performance, leading to three specific failure modes:
    \vspace{0.3cm}
    \begin{columns}
        \column{0.33\textwidth}
        \centering
        \includegraphics[width=\linewidth, height=0.8\linewidth]{../report/img/output/det_000018.jpg}
        \footnotesize{\textbf{Fabrication:} Detecting non-existent objects.}

        \column{0.33\textwidth}
        \centering
        \includegraphics[width=\linewidth, height=0.8\linewidth]{../report/img/output/det_000049.jpg}
        \footnotesize{\textbf{Mislabeling:} Predicting wrong classes.}

        \column{0.33\textwidth}
        \centering
        \includegraphics[width=\linewidth, height=0.8\linewidth]{../report/img/009909_0.13.jpg}
        \footnotesize{\textbf{Vanishing:} Valid objects erased from prediction.}
    \end{columns}
    
    \vspace{0.3cm}
    \textbf{Transferability:} Attacks generated for YOLO11n also degraded SSDlite (Black-box scenario), confirming the universality of the vulnerability.
\end{frame}

\subsection{Defense Results}
\begin{frame}{Adversarial Training: The Trade-off}
    We fine-tuned YOLO11n on adversarial examples.
    \vspace{0.3cm}
    
    \begin{columns}
        \column{0.4\textwidth}
        \includegraphics[width=\linewidth]{../report/img/advtrained.png}
        
        \column{0.6\textwidth}
        \textbf{Key Findings:}
        \begin{itemize}
            \item \textbf{Trade-off:} Increasing $\varepsilon$ improves robustness but degrades performance on clean images.
            \item \textbf{Low $\varepsilon$ (0.02):} Negligible defense.
            \item \textbf{High $\varepsilon$ (0.20):} Destroys clean accuracy.
            \item \textbf{Sweet Spot:} We identified $\varepsilon \in [0.08, 0.10]$ as the optimal balance for this architecture.
        \end{itemize}
    \end{columns}
\end{frame}

\subsection{Critique of Imperceptibility}
\begin{frame}{Critique of Imperceptibility}
    Is $L_\infty$ a good proxy for human perception?
    
    \begin{columns}
        \column{0.4\textwidth}
        \includegraphics[width=\linewidth]{../report/img/worsedist.png}
        
        \column{0.6\textwidth}
        \begin{itemize}
            \item We compared $L_\infty$ against perceptual metrics: \textbf{SSIM} and \textbf{LPIPS}.
            \item \textbf{Result:} While $L_\infty$ scales linearly with $\varepsilon$, perceptual degradation (LPIPS) is non-linear.
            \item \textbf{Implication:} Future defenses should optimize against perceptual distances rather than simple pixel norms.
        \end{itemize}
    \end{columns}
\end{frame}


%%%%%%%%%%%%%%%%%%%%%%%%%%%%%
% CONCLUSION
%%%%%%%%%%%%%%%%%%%%%%%%%%%%%
\section{Conclusion}
\subsection{Conclusion}
\begin{frame}{Conclusion}
    \begin{enumerate}
        \item \textbf{Validation:} We confirmed Goodfellow's hypothesis that linearity is the primary cause of adversarial vulnerability.
        \item \textbf{Generalization:} We showed these vulnerabilities persist in complex Object Detection tasks (YOLO11n).
        \item \textbf{Defense:} Adversarial Training is effective but introduces a critical trade-off between robustness and standard accuracy.
        \item \textbf{Future Work:} Investigating patch-based attacks and integrating perceptual metrics (LPIPS) into the loss function.
    \end{enumerate}
\end{frame}

%%%%%%%%%%%%%%%%%
% REFERENCES
%%%%%%%%%%%%%%%%%
\subsection{References}
\begin{frame}[allowframebreaks]{Bibliography}
    \frametitle{Bibliography}
    \printbibliography[heading=none]
\end{frame}

%%%%%%%%%%%%%%%%%
% ANNEX
%%%%%%%%%%%%%%%%%
% \subsection{Annex}
% \begin{frame}{Annex: Section}
%     \frametitle{Annex title}
%     
% \end{frame}

\end{document}
